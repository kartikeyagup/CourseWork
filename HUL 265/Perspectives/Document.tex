\documentclass{article}
\usepackage[utf8]{inputenc}
% \documentclass{beamer}

\title{\textbf{Report on different perspectives}
	  \\ {\large \textbf{HUL265: Theories of Personality}}}
\author
{
	\textbf{Kartikeya Gupta}\\ 
	2013CS10231\\
}

\date{19th April 2015}



\begin{document}
	\maketitle
	\tableofcontents
	\newpage
	\section{Phsychoanalysis}
		\subsection{About}
			This was the first perspective in psychology. The authors were Freud, Carl Jung, Adler.
		\subsection{Research Methods}
			\begin{itemize}
				\item Free Association
				\item Dream analysis, Jung's analysis of dreams
				\item Projective Techniques
				\item Word Association
				\item Psychological tests
			\end{itemize}
		\subsection{Abnormal Characteristics}
			\begin{itemize}
				\item Freud had categorized the life into different stages. On not being able to complete a stage, one gets fixated in that and shows abnormal behavior.
			\end{itemize}
		\subsection{Normal Characteristics}
			\begin{itemize}
				\item From Freud's perspective, successfully completing all the stages lead to a normal personality.
			\end{itemize}
		\subsection{Therapies}
			\begin{itemize}
				\item Bringing memories from the unconscious to the conscious and thereby freeing a person from its effects.
				\item Psychoanalytic therapy.
			\end{itemize}
		\subsection{Analysis}
			\subsubsection{Positive}
				\begin{itemize}
					\item Freud's theory was the first psychological theory and psychotherapy.
					\item Concepts like, Id, Ego, Superego, Conscious, Pre-Conscious, Unconscious.
					\item It had the importance of first memories and childhood experiences.
					\item The importance of self actualisation had come in Jung's theory
					\item Adler gave importance to the presence of social variables.
					\item The concept of defence mechanisms 
				\end{itemize}
			\subsubsection{Negative}
				\begin{itemize}
					\item Therapy is very time-consuming and is unlikely to provide answers quickly.
					\item Lack of scientific explanation of many terms. Libido could not be measured and the time between cause and effect was too long to understand the relationships between variables.
					\item Poor research was there in Freuds theory. The concept of collective unconscious in Jung's theory was unscientific.
				\end{itemize}

	\section{Behaviorism}
		\subsection{About}
			Watson, Pavlov and Skinner were the main contributors in this area.
		\subsection{Research Methods}
			\begin{itemize}
				\item Experimental Method
				\item Quasi Experimental Method
			\end{itemize}
		\subsection{Abnormal Characteristics}
			\begin{enumerate}
				\item Improper environment conditioning
				\item Lack of reinforcement or too much punishment
			\end{enumerate}
		\subsection{Normal Characteristics}
			A mix of both positive and negative reinforcement and proper environment conditions lead to normal behavior characteristics. Proper conditioning is also important.
		\subsection{Therapies}
			\begin{itemize}
				\item Flooding
				\item Exposure therapy
				\item Systematic desensitization
			\end{itemize}
		\subsection{Analysis}
			\subsubsection{Positive}
				\begin{itemize}
					\item The entire approach was very scientific.
					\item It gave a lot of scientific explanations for human behavior.
					\item Lots of experiments backed up the proposed theories.
					\item Led to a lot of research in reinforcement and punishment.
					\item Gave importance to stimulus response relations.
				\end{itemize}
			\subsubsection{Negative}
				\begin{itemize}
					\item There was excessive generalization from animals to humans in terms of experiments and theories.
					\item It does not account for all types of learning as the activities of the mind are not given attention to.
					\item Very little freedom of interpretation is given as this is very deterministic.
				\end{itemize}

	\section{Cognitive}
		\subsection{About}
			Wundt, Kohler, Tolman and Seligman were main contributors of this theory.
		\subsection{Research Methods}
			\begin{itemize}
				\item Observation Method
				\item Participatory action research
				\item Introspection method
				\item Interviews
			\end{itemize}
		\subsection{Therapies}
			There are many cognitive therapies like the Rational Emotive Therapy by Albert Ellis and Beck's Cognitive Behaviour therapy for depression.
			\newline
			These therapies change pessimistic ideas, unreal expectations and overly critical self evaluations which create depression.
		\subsection{Abnormal Characteristics}
			\begin{itemize}
				\item Having lots of pessimistic ideas.
				\item Unrealistic expectations.
				\item Overly critical self evaluations.
				\item Depression.
				\item Differences between real and ideal self.
			\end{itemize}
		\subsection{Normal Characteristics}
			\begin{itemize}
				\item Positive thoughts and ideas.
				\item Normal self evaluations.
				\item Having a happy life.
			\end{itemize}
		\subsection{Analysis}
			\subsubsection{Positive}
				\begin{itemize}
					\item It values the thoughts and cognitive processes of an individual. These were previously ignored by all perspectives.
				\end{itemize}
			\subsubsection{Negative}
				\begin{itemize}
					\item This theory does not distinguish a human from a computer. All the processes are assumed to be done like a machine which is incorrect as humans get affected by emotions and other thoughts.
					\item There is no linking between the mind and other parts in this theory. This leads to incorrect results as humans are not only controlled by their mind but by other organs as well.
				\end{itemize}

	\section{Socio-Cognitive}
		\subsection{About}
		Bandura was the main contributor to this.
		He linked behaviorism and cognitive perspectives together.
		\newline
		He gave a lot of importance on self efficacy and esteem and on different types of reinforcement.
		\subsection{Research Methods}
			\begin{itemize}
				\item Experimental method
				\item Observational method
			\end{itemize}
		\subsection{Therapies}
			Therapies include Self control therapy and modeling therapy.
			\newline
			In Self control therapy, one makes charts of behavior and does planning based on the environment. Then one sets up plans to reward and punish oneself based on the adhering or not of plans.
		\subsection{Abnormal Characteristics}
			\begin{itemize}
				\item Improper conditioning
				\item Lack of belief in oneself
				\item Discrepancy of real and ideal self which one perceives and what one really is
			\end{itemize}
		\subsection{Normal Characteristics}
			\begin{itemize}
				\item Proper environment and conditioning
				\item Self efficacy
			\end{itemize}
		\subsection{Analysis}
			\subsubsection{Positive}
				\begin{itemize}
					\item Very scientific
					\item Leads to a lot of emphasis on self and cognitive processes.
					\item Has a lot of important implications in the daily lives.
					\item Led to explanation of self efficacy and explained more about reinforcement.
				\end{itemize}
			\subsubsection{Negative}
				\begin{itemize}
					\item Bandura's theory has undermined the role of genetic and maturational variables.
					\item The theory is overall not very developmental.
				\end{itemize}

	\section{Types and Traits}
		\subsection{About}
			Gordon Allport, Raymond B Catell and Henry Eyesenck were the main contributors to this perspective.
		\subsection{Research Methods}
			\begin{itemize}
				\item Interviews
				\item Observations on behavior
			\end{itemize}
		\subsection{Therapies}
			\begin{itemize}
				\item Flooding
				\item Dream analysis
				\item Psychoanalytic therapy
			\end{itemize}
		\subsection{Abnormal Characteristics}
			\begin{itemize}
				\item Not able to experience affection for others.
				\item Below average emotional control, will-power, slow thoughts.
				\item Having very high emotional reactivity which leads to phobias, compulsions and obsessions.
				\item Being very emotionally reactive with low cortical excitation leading to psychopathic nature.
			\end{itemize}
		\subsection{Normal Characteristics}
			\begin{itemize}
				\item Self extension
				\item Realistic perception
				\item Self objectification
				\item Capacity for warm human interactions
			\end{itemize}
		\subsection{Analysis}
			\subsubsection{Positive}
				\begin{itemize}
					\item There were lots of original concepts and methodologies.
					\item Explanations about introversion and extraversion.
					\item Gave 5 personality factors OCEAN which is Openness, Conscientiousness, Extraversion, Agreeableness, Neuroticism.

				\end{itemize}
			\subsubsection{Negative}
				\begin{itemize}
					\item Allport's theory lacked scientific rigor.
					\item There was lack of theory and circularity was present in Allport's work.
					\item Catell's work was too subjective.
					\item Catell put too much emphasis on averages and groups.
					\item Lack of explanation on why individuals behave in a certain manner.
				\end{itemize}

	\section{Humanistic}
		\subsection{About}
			Rogers and Maslow are the main contributors of this perspective.
		\subsection{Research Methods}
			\begin{itemize}
				\item Q Sort Technique
				\item Questionnaires
				\item Interviews
			\end{itemize}
		\subsection{Therapies}
			\begin{itemize}
				\item Reflection of oneself.
				\item Client Centered therapy
				\item Non directive therapy
				\item Rogerian therapy
			\end{itemize}
		\subsection{Abnormal Characteristics}
			\begin{itemize}
				\item Anxiety
				\item Incongruence between real self and ideal self
				\item Conditional positive regard which leads to self discrepancies.
			\end{itemize}
		\subsection{Normal Characteristics}
			\begin{itemize}
				\item Having a tendency for self actualization.
				\item Positive self regard
				\item Unconditional positive regard
				\item Openness to experience
				\item Creativity
			\end{itemize}
		\subsection{Analysis}
			\subsubsection{Positive}
				\begin{itemize}
					\item Led to concept of conscious experience which is a a sense of moving through space and time. This is something that is makes us human.
					\item It gives an overall perspective of the human nature and explains a lot of things like behavior tendencies.
				\end{itemize}
			\subsubsection{Negative}
				\begin{itemize}
					\item Traits are not able to be predicted from this.
					\item Abilities and interest development is not mentioned.
					\item Conscious experience from Maslow is subjective.
					\item Less scientific study and research
				\end{itemize}
\end{document}