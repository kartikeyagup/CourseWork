\documentclass{article}
\usepackage[utf8]{inputenc}
% \documentclass{beamer}

\title{\textbf{Report on Research Papers}
	  \\ {\large \textbf{HUL265: Theories of Personality}}}
\author
{
	\textbf{Kartikeya Gupta}\\ 
	2013CS10231\\
}

\date{12th April 2015}



\begin{document}
	\maketitle

	\section{Creative mood swings: divergent and convergent thinking affect mood in opposite ways}
		\subsection{Author:} Soghra Akbari Chermahini, Bernhard Hommel
		\subsection{Date of publishing:} 22 Jun 2011
		\subsection{Link:} http://link.springer.com/article/10.1007/s00426-011-0358-z
		\subsection{Why did I choose this paper?}
			I have observed in me and my friends how our mood varies from time to time. It depends on what activities we are involved in and affects the way we work. If a person is in a good mood then he spreads positivity. If he is upset then it creates a negative atmosphere for others. I was curious on knowing more about this so that I can understand it better.
		\subsection{What did I learn from this paper?}
			\begin{itemize}
				\item Being in good or bad mood is dependent on the thoughts and the state of mind of a person
				\item People get more motivated by tasks that are difficult but solvable than by easy tasks. This is based on what Skinner said that one should break tasks into smaller tasks and look for positive reinforcement.
				\item Combining high motivation and success is associated with positive mood as participants in the experiment showed more positive mood after performing the convergent-thinking task as compared with the divergent thinking task.
				\item Reward leads to positive affect. Identifying a correct response in a more difficult task is more rewarding and hence brings more positive mood than doing so in an easier task. Hence people should try to get break the tasks into smaller goals so that they receive more positive reinforcement.
			\end{itemize}

	\section{Context-dependent motor skill and the role of practice}
		\subsection{Author:} Marit F. L. Ruitenberg, Elian De Kleine, Rob H. J. Van der Lubbe, Willem B. Verwey, Elger L. Abrahamse
		\subsection{Date of publishing:} 24 Oct 2011
		\subsection{Link:} http://link.springer.com/article/10.1007/s00426-011-0388-6
		\subsection{Why did I choose this paper?}
			I wanted to learn about the role of practice. We are often told that practice makes a man perfect. In my opinion practice does help us in improving skills but I wanted to know what research says about the same. Hence I chose this paper. 
		\subsection{What did I learn from this paper?}
			\begin{itemize}
				\item I learned about ANOVA test (Analysis of variance test). In this a collection of statistical models used in order to analyze the differences between group means and procedures. This is used to replace doing multiple T Tests as there is lesser possibility of errors in this case. Multiple T tests leads to severe complications.
				\item From the results of the paper, I came to know that practice doesn't always change the performance of individuals. If the task is intuitive then practice has little or no effect as memory retrieval is not needed. Whereas if a task is based on the previous done activities, then practice can bring a considerable change in the performance in that activity.
				\item The effects of practice are task-specific. For tasks in which the stimulus remains essential even after extensive practice there is increasingly stronger effects of context change with practice. All tasks are a mixture of different types of sub tasks. So in each activity a role of practice is bound to come in.
				\item Even after these experiments, the role of practice cannot me completely undermined as it has been proved to have some effect in certain cases. The paper has shown that practice alone cannot solely affect the performance in activities. To maximize the output, one should optimize on practice and on intuitive activities.
			\end{itemize} 
	
	\section{Effects of exposure to facial expression variation in face learning and recognition}
		\subsection{Author:} Chang Hong Liu, Wenfeng Chen, James Ward
		\subsection{Date of publishing:} 6 Nov 2014
		\subsection{Link:} http://link.springer.com/article/10.1007/s00426-014-0627-8
		\subsection{Why did I choose this paper?}
			In lots of news articles I have read about people having eidetic memory. It means being able to remember things which they see easily. I have been fascinated by this skill which people have and wanted to know more about it. Some people are able to understand the facial expressions of others better and able to understand the mood and what is going on in the mind of the individual just by taking a look at his face. This skill is very appealing to me and I wanted to know more about it.
		\subsection{What did I learn from this paper?}
			\begin{itemize}
				\item ANOVA test was used in this paper in one out of the 3 tests. In the 2nd test the standard deviation for the data from 2 different target audiences was used. In the 3rd test a mixture of the 2 tests was used so that better understanding of the individual components affecting facial expression recognition can be studied. 
				\item From experiment 1 the results showed no direct indication that exposure to a single expression showed any change. This means that we cannot directly implicate an expression. But when multiple expressions were showed, a person can distinguish between them easier.
				\item From experiment 2 the results showed that when there are multiple expressions, the person is able to relate to one of them more than the others which leads to people being able to distinguish an expression from the other. 
				\item From the overall experiments, I came to know that certain expressions have better recognition than others. Like a happy face is more relatable to than a blank face. An angry face is more relatable to a confused face. This is because a person sees more happy faces than blank and more angry ones than confused and hence is able to distinguish to a greater extent.
				\item I also learned from this paper that the reason we are able to understand another persons mood from taking a look at his face is because we are able to relate it to some previous expressions we have seen. This means that with more time and interaction, we can understand the mood of a person as we will be able to relate the facial expressions with some previous ones.
			\end{itemize}

\end{document}