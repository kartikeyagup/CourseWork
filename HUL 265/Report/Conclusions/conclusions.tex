\def\baselinestretch{1}
\chapter{Conclusions}
\ifpdf
    \graphicspath{{Conclusions/ConclusionsFigs/PNG/}{Conclusions/ConclusionsFigs/PDF/}{Conclusions/ConclusionsFigs/}}
\else
    \graphicspath{{Conclusions/ConclusionsFigs/EPS/}{Conclusions/ConclusionsFigs/}}
\fi

\def\baselinestretch{1.66}

Synergy Sansthan is a very young and interesting NGO. It has a different ideology as compared from other NGOs. It has produced a lot of results as well.

\vspace{1cm}

The most striking fact that I liked was the policy of the creators that they would all leave the NGO at the age of 35 and not stay on the board or be in any official way related to it. This will ensure that the NGO stays an NGO for the youth and takes more and more action. It will not become a dynastic NGO which is run for profit.


\vspace{1cm}
As it is targeting the youth, it is able to deliver maximum amount of change. By organizing workshops in colleges in Harda the issues of the youth are directly faced and resolved.

\vspace{1cm}

Synergy Sansthan should be given more opportunities and allowed to grow more. Its scope right now is limited to a few districts in Madhya Pradesh. If this increases then the amount of change that it can deliver will be monumental.
%%% ----------------------------------------------------------------------

% ------------------------------------------------------------------------

%%% Local Variables: 
%%% mode: latex
%%% TeX-master: "../thesis"
%%% End: 
